% Options for packages loaded elsewhere
\PassOptionsToPackage{unicode}{hyperref}
\PassOptionsToPackage{hyphens}{url}
%
\documentclass[
]{article}
\usepackage{amsmath,amssymb}
\usepackage{iftex}
\ifPDFTeX
  \usepackage[T1]{fontenc}
  \usepackage[utf8]{inputenc}
  \usepackage{textcomp} % provide euro and other symbols
\else % if luatex or xetex
  \usepackage{unicode-math} % this also loads fontspec
  \defaultfontfeatures{Scale=MatchLowercase}
  \defaultfontfeatures[\rmfamily]{Ligatures=TeX,Scale=1}
\fi
\usepackage{lmodern}
\ifPDFTeX\else
  % xetex/luatex font selection
\fi
% Use upquote if available, for straight quotes in verbatim environments
\IfFileExists{upquote.sty}{\usepackage{upquote}}{}
\IfFileExists{microtype.sty}{% use microtype if available
  \usepackage[]{microtype}
  \UseMicrotypeSet[protrusion]{basicmath} % disable protrusion for tt fonts
}{}
\makeatletter
\@ifundefined{KOMAClassName}{% if non-KOMA class
  \IfFileExists{parskip.sty}{%
    \usepackage{parskip}
  }{% else
    \setlength{\parindent}{0pt}
    \setlength{\parskip}{6pt plus 2pt minus 1pt}}
}{% if KOMA class
  \KOMAoptions{parskip=half}}
\makeatother
\usepackage{xcolor}
\usepackage[margin=1in]{geometry}
\usepackage{color}
\usepackage{fancyvrb}
\newcommand{\VerbBar}{|}
\newcommand{\VERB}{\Verb[commandchars=\\\{\}]}
\DefineVerbatimEnvironment{Highlighting}{Verbatim}{commandchars=\\\{\}}
% Add ',fontsize=\small' for more characters per line
\usepackage{framed}
\definecolor{shadecolor}{RGB}{248,248,248}
\newenvironment{Shaded}{\begin{snugshade}}{\end{snugshade}}
\newcommand{\AlertTok}[1]{\textcolor[rgb]{0.94,0.16,0.16}{#1}}
\newcommand{\AnnotationTok}[1]{\textcolor[rgb]{0.56,0.35,0.01}{\textbf{\textit{#1}}}}
\newcommand{\AttributeTok}[1]{\textcolor[rgb]{0.13,0.29,0.53}{#1}}
\newcommand{\BaseNTok}[1]{\textcolor[rgb]{0.00,0.00,0.81}{#1}}
\newcommand{\BuiltInTok}[1]{#1}
\newcommand{\CharTok}[1]{\textcolor[rgb]{0.31,0.60,0.02}{#1}}
\newcommand{\CommentTok}[1]{\textcolor[rgb]{0.56,0.35,0.01}{\textit{#1}}}
\newcommand{\CommentVarTok}[1]{\textcolor[rgb]{0.56,0.35,0.01}{\textbf{\textit{#1}}}}
\newcommand{\ConstantTok}[1]{\textcolor[rgb]{0.56,0.35,0.01}{#1}}
\newcommand{\ControlFlowTok}[1]{\textcolor[rgb]{0.13,0.29,0.53}{\textbf{#1}}}
\newcommand{\DataTypeTok}[1]{\textcolor[rgb]{0.13,0.29,0.53}{#1}}
\newcommand{\DecValTok}[1]{\textcolor[rgb]{0.00,0.00,0.81}{#1}}
\newcommand{\DocumentationTok}[1]{\textcolor[rgb]{0.56,0.35,0.01}{\textbf{\textit{#1}}}}
\newcommand{\ErrorTok}[1]{\textcolor[rgb]{0.64,0.00,0.00}{\textbf{#1}}}
\newcommand{\ExtensionTok}[1]{#1}
\newcommand{\FloatTok}[1]{\textcolor[rgb]{0.00,0.00,0.81}{#1}}
\newcommand{\FunctionTok}[1]{\textcolor[rgb]{0.13,0.29,0.53}{\textbf{#1}}}
\newcommand{\ImportTok}[1]{#1}
\newcommand{\InformationTok}[1]{\textcolor[rgb]{0.56,0.35,0.01}{\textbf{\textit{#1}}}}
\newcommand{\KeywordTok}[1]{\textcolor[rgb]{0.13,0.29,0.53}{\textbf{#1}}}
\newcommand{\NormalTok}[1]{#1}
\newcommand{\OperatorTok}[1]{\textcolor[rgb]{0.81,0.36,0.00}{\textbf{#1}}}
\newcommand{\OtherTok}[1]{\textcolor[rgb]{0.56,0.35,0.01}{#1}}
\newcommand{\PreprocessorTok}[1]{\textcolor[rgb]{0.56,0.35,0.01}{\textit{#1}}}
\newcommand{\RegionMarkerTok}[1]{#1}
\newcommand{\SpecialCharTok}[1]{\textcolor[rgb]{0.81,0.36,0.00}{\textbf{#1}}}
\newcommand{\SpecialStringTok}[1]{\textcolor[rgb]{0.31,0.60,0.02}{#1}}
\newcommand{\StringTok}[1]{\textcolor[rgb]{0.31,0.60,0.02}{#1}}
\newcommand{\VariableTok}[1]{\textcolor[rgb]{0.00,0.00,0.00}{#1}}
\newcommand{\VerbatimStringTok}[1]{\textcolor[rgb]{0.31,0.60,0.02}{#1}}
\newcommand{\WarningTok}[1]{\textcolor[rgb]{0.56,0.35,0.01}{\textbf{\textit{#1}}}}
\usepackage{longtable,booktabs,array}
\usepackage{calc} % for calculating minipage widths
% Correct order of tables after \paragraph or \subparagraph
\usepackage{etoolbox}
\makeatletter
\patchcmd\longtable{\par}{\if@noskipsec\mbox{}\fi\par}{}{}
\makeatother
% Allow footnotes in longtable head/foot
\IfFileExists{footnotehyper.sty}{\usepackage{footnotehyper}}{\usepackage{footnote}}
\makesavenoteenv{longtable}
\usepackage{graphicx}
\makeatletter
\def\maxwidth{\ifdim\Gin@nat@width>\linewidth\linewidth\else\Gin@nat@width\fi}
\def\maxheight{\ifdim\Gin@nat@height>\textheight\textheight\else\Gin@nat@height\fi}
\makeatother
% Scale images if necessary, so that they will not overflow the page
% margins by default, and it is still possible to overwrite the defaults
% using explicit options in \includegraphics[width, height, ...]{}
\setkeys{Gin}{width=\maxwidth,height=\maxheight,keepaspectratio}
% Set default figure placement to htbp
\makeatletter
\def\fps@figure{htbp}
\makeatother
\setlength{\emergencystretch}{3em} % prevent overfull lines
\providecommand{\tightlist}{%
  \setlength{\itemsep}{0pt}\setlength{\parskip}{0pt}}
\setcounter{secnumdepth}{5}
\usepackage{booktabs}
\usepackage{longtable}
\usepackage{array}
\usepackage{multirow}
\usepackage{wrapfig}
\usepackage{float}
\usepackage{colortbl}
\usepackage{pdflscape}
\usepackage{tabu}
\usepackage{threeparttable}
\usepackage{threeparttablex}
\usepackage[normalem]{ulem}
\usepackage{makecell}
\usepackage{xcolor}
\ifLuaTeX
  \usepackage{selnolig}  % disable illegal ligatures
\fi
\IfFileExists{bookmark.sty}{\usepackage{bookmark}}{\usepackage{hyperref}}
\IfFileExists{xurl.sty}{\usepackage{xurl}}{} % add URL line breaks if available
\urlstyle{same}
\hypersetup{
  pdftitle={trcpetc R package},
  pdfauthor={Jessica Weiss},
  hidelinks,
  pdfcreator={LaTeX via pandoc}}

\title{trcpetc R package}
\author{Jessica Weiss}
\date{2025-11-04}

\begin{document}
\maketitle

{
\setcounter{tocdepth}{3}
\tableofcontents
}
\hypertarget{welcome-to-trcpetc}{%
\section{Welcome to trcpetc}\label{welcome-to-trcpetc}}

This vignette is designed to help you get acquainted with the new functionality introduced in the \texttt{trcpetc} package, which is an update to the previously used \texttt{fanetc} package.

In addition to new functionality, the package includes a sample dataset called \texttt{cardio\_data}, along with a corresponding data dictionary named \texttt{cardio\_data\_dictionary}. All examples in this vignette use this dataset as a consistent reference point.

\hypertarget{descriptive-tables}{%
\section{Descriptive tables}\label{descriptive-tables}}

\hypertarget{pre-processing}{%
\subsection{Pre-processing}\label{pre-processing}}

\begin{itemize}
\item
  Before using \texttt{table\_one()}, there are now two functions available for pre-processing data, allowing you to display it in a way that may be more interpretable in the final table.
\item
  These include the functions \texttt{factor\_order()} and \texttt{check\_box\_convert()}
\end{itemize}

\hypertarget{factor_order}{%
\subsubsection{\texorpdfstring{\texttt{factor\_order()}}{factor\_order()}}\label{factor_order}}

\begin{itemize}
\item
  Used to sort a factor variable by descending frequency.
\item
  The example below is comparing the output of \texttt{table\_one()} with and without \texttt{factor\_order()}.
\end{itemize}

\begin{Shaded}
\begin{Highlighting}[]
\CommentTok{\# Without factor\_order}
\NormalTok{cardio\_data }\SpecialCharTok{\%\textgreater{}\%} \FunctionTok{select}\NormalTok{(SurgeryType) }\SpecialCharTok{\%\textgreater{}\%} \FunctionTok{table\_one}\NormalTok{()}
\end{Highlighting}
\end{Shaded}

\begin{table}
\centering
\begin{tabular}[t]{lcc}
\toprule
\multicolumn{1}{c}{} & \multicolumn{2}{c}{Overall (N = 185)} \\
\cmidrule(l{3pt}r{3pt}){2-3}
\multicolumn{1}{c}{Variables} & \multicolumn{1}{c}{N} & \multicolumn{1}{c}{Stat}\\
\midrule
\textbf{SurgeryType} & 185 & \\
\hspace{1em}Aortic repair &  & 9 (5%)\\
\hspace{1em}Aortic replacement &  & 7 (4%)\\
\hspace{1em}CABG &  & 50 (27%)\\
\hspace{1em}LVAD &  & 8 (4%)\\
\addlinespace
\hspace{1em}Maze &  & 6 (3%)\\
\hspace{1em}Mitral repair &  & 20 (11%)\\
\hspace{1em}Mitral replacement &  & 77 (42%)\\
\hspace{1em}Transplant &  & 8 (4%)\\
\bottomrule
\end{tabular}
\end{table}

\begin{Shaded}
\begin{Highlighting}[]
\CommentTok{\# With factor\_order}
\NormalTok{cardio\_data }\SpecialCharTok{\%\textgreater{}\%} 
  \FunctionTok{mutate}\NormalTok{(}\AttributeTok{SurgeryType =} \FunctionTok{factor\_order}\NormalTok{(SurgeryType)) }\SpecialCharTok{\%\textgreater{}\%} \FunctionTok{select}\NormalTok{(SurgeryType) }\SpecialCharTok{\%\textgreater{}\%} \FunctionTok{table\_one}\NormalTok{()}
\end{Highlighting}
\end{Shaded}

\begin{table}
\centering
\begin{tabular}[t]{lcc}
\toprule
\multicolumn{1}{c}{} & \multicolumn{2}{c}{Overall (N = 185)} \\
\cmidrule(l{3pt}r{3pt}){2-3}
\multicolumn{1}{c}{Variables} & \multicolumn{1}{c}{N} & \multicolumn{1}{c}{Stat}\\
\midrule
\textbf{SurgeryType} & 185 & \\
\hspace{1em}Mitral replacement &  & 77 (42%)\\
\hspace{1em}CABG &  & 50 (27%)\\
\hspace{1em}Mitral repair &  & 20 (11%)\\
\hspace{1em}Aortic repair &  & 9 (5%)\\
\addlinespace
\hspace{1em}LVAD &  & 8 (4%)\\
\hspace{1em}Transplant &  & 8 (4%)\\
\hspace{1em}Aortic replacement &  & 7 (4%)\\
\hspace{1em}Maze &  & 6 (3%)\\
\bottomrule
\end{tabular}
\end{table}

\hypertarget{check_box_convert}{%
\subsubsection{\texorpdfstring{\texttt{check\_box\_convert()}}{check\_box\_convert()}}\label{check_box_convert}}

\begin{itemize}
\item
  This function is used to format columns that are part of a single ``check box'' question.
\item
  It accepts a list of column names (check\_box\_cols) as input. These columns must contain either logical (TRUE/FALSE) values or binary indicators (0/1).
\item
  The function converts these columns to logical format and assumes that if none of the options are selected (i.e., all values are FALSE or 0), then all should be treated as missing (NA).
\item
  The optional title argument allows you to create a new column that serves as a header for the checkbox group. This column will be TRUE for rows with at least one selected option and NA otherwise.
\item
  In the table\_one() function, this information can be incorporated using the check\_box argument for the column names and check\_box\_title for the title names. This ensures proper formatting and allows multiple checkbox groups and titles to be included.
\item
  I recommend adding a footnote to clarify that percentages may not sum to 100\%, as respondents can select multiple options.
\item
  See example below with and without applying \texttt{check\_box\_convert()}, and an example with multiple check box questions
\end{itemize}

\begin{Shaded}
\begin{Highlighting}[]
\NormalTok{Comorbidities  }\OtherTok{\textless{}{-}}\NormalTok{ cardio\_data }\SpecialCharTok{\%\textgreater{}\%} \FunctionTok{select}\NormalTok{(Diabetes}\SpecialCharTok{:}\NormalTok{CAD) }\SpecialCharTok{\%\textgreater{}\%} \FunctionTok{names}\NormalTok{()}

\DocumentationTok{\#\# Without checkbox question (Assuming missing all questions is FALSE)}
\NormalTok{cardio\_data }\SpecialCharTok{\%\textgreater{}\%} \FunctionTok{select}\NormalTok{(Comorbidities) }\SpecialCharTok{\%\textgreater{}\%} 
   \FunctionTok{table\_one}\NormalTok{()}
\end{Highlighting}
\end{Shaded}

\begin{table}
\centering
\begin{tabular}[t]{lcc}
\toprule
\multicolumn{1}{c}{} & \multicolumn{2}{c}{Overall (N = 185)} \\
\cmidrule(l{3pt}r{3pt}){2-3}
\multicolumn{1}{c}{Variables} & \multicolumn{1}{c}{N} & \multicolumn{1}{c}{Stat}\\
\midrule
\textbf{Diabetes} & 185 & 33 (18%)\\
\textbf{Hypertension} & 185 & 35 (19%)\\
\textbf{COPD} & 185 & 41 (22%)\\
\textbf{CKD} & 185 & 39 (21%)\\
\textbf{CAD} & 185 & 38 (21%)\\
\bottomrule
\end{tabular}
\end{table}

\begin{Shaded}
\begin{Highlighting}[]
\CommentTok{\# Example with one checkbox question (Assumes missing all questions is missing)}

\NormalTok{cardio\_data }\SpecialCharTok{\%\textgreater{}\%} \FunctionTok{select}\NormalTok{(Comorbidities) }\SpecialCharTok{\%\textgreater{}\%} 
  \FunctionTok{check\_box\_convert}\NormalTok{(}\AttributeTok{check\_box\_cols =}\NormalTok{ Comorbidities,}\AttributeTok{title =} \StringTok{"Comorbidities¹"}\NormalTok{)  }\SpecialCharTok{\%\textgreater{}\%} 
  \FunctionTok{table\_one}\NormalTok{(}\AttributeTok{Check\_box =}\NormalTok{ Comorbidities,}
            \AttributeTok{Check\_box\_title =} \StringTok{"Comorbidities¹"}\NormalTok{)}\SpecialCharTok{\%\textgreater{}\%}
\NormalTok{  kableExtra}\SpecialCharTok{::}\FunctionTok{footnote}\NormalTok{(}
    \AttributeTok{general =} \StringTok{"¹Patients could present with more than one comorbidity, totals may not sum to 100\%."}\NormalTok{,}
    \AttributeTok{general\_title =} \StringTok{""}\NormalTok{,}
    \AttributeTok{footnote\_as\_chunk =} \ConstantTok{TRUE}
\NormalTok{  ) }
\end{Highlighting}
\end{Shaded}

\begin{table}
\centering
\begin{tabular}[t]{lcc}
\toprule
\multicolumn{1}{c}{} & \multicolumn{2}{c}{Overall (N = 185)} \\
\cmidrule(l{3pt}r{3pt}){2-3}
\multicolumn{1}{c}{Variables} & \multicolumn{1}{c}{N} & \multicolumn{1}{c}{Stat}\\
\midrule
\textbf{Comorbidities¹} & 128 & \\
\hspace{1em}Diabetes &  & 33 (26%)\\
\hspace{1em}Hypertension &  & 35 (27%)\\
\hspace{1em}COPD &  & 41 (32%)\\
\hspace{1em}CKD &  & 39 (30%)\\
\addlinespace
\hspace{1em}CAD &  & 38 (30%)\\
\bottomrule
\multicolumn{3}{l}{\rule{0pt}{1em}¹Patients could present with more than one comorbidity, totals may not sum to 100\%.}\\
\end{tabular}
\end{table}

\begin{Shaded}
\begin{Highlighting}[]
\DocumentationTok{\#\# Example with two different checkbox questions}

\NormalTok{Comorbidities1  }\OtherTok{\textless{}{-}}\NormalTok{ cardio\_data }\SpecialCharTok{\%\textgreater{}\%} \FunctionTok{select}\NormalTok{(Diabetes}\SpecialCharTok{:}\NormalTok{COPD) }\SpecialCharTok{\%\textgreater{}\%} \FunctionTok{names}\NormalTok{()}
\NormalTok{Comorbidities2  }\OtherTok{\textless{}{-}}\NormalTok{ cardio\_data }\SpecialCharTok{\%\textgreater{}\%} \FunctionTok{select}\NormalTok{(CKD}\SpecialCharTok{:}\NormalTok{CAD) }\SpecialCharTok{\%\textgreater{}\%} \FunctionTok{names}\NormalTok{()}



\NormalTok{cardio\_data }\SpecialCharTok{\%\textgreater{}\%} \FunctionTok{select}\NormalTok{(Comorbidities1,Comorbidities2) }\SpecialCharTok{\%\textgreater{}\%} 
  \FunctionTok{check\_box\_convert}\NormalTok{(}\AttributeTok{check\_box\_cols =}\NormalTok{ Comorbidities1,}\AttributeTok{title =} \StringTok{"Comorbidities1¹"}\NormalTok{)  }\SpecialCharTok{\%\textgreater{}\%} 
  \FunctionTok{check\_box\_convert}\NormalTok{(}\AttributeTok{check\_box\_cols =}\NormalTok{ Comorbidities2,}\AttributeTok{title =} \StringTok{"Comorbidities2¹"}\NormalTok{)  }\SpecialCharTok{\%\textgreater{}\%} 
  \FunctionTok{table\_one}\NormalTok{(}\AttributeTok{Check\_box =}\NormalTok{ Comorbidities,}
            \AttributeTok{Check\_box\_title =} \FunctionTok{c}\NormalTok{(}\StringTok{"Comorbidities¹"}\NormalTok{,}\StringTok{"Comorbidities2¹"}\NormalTok{))}\SpecialCharTok{\%\textgreater{}\%}
\NormalTok{  kableExtra}\SpecialCharTok{::}\FunctionTok{footnote}\NormalTok{(}
    \AttributeTok{general =} \StringTok{"¹Patients could present with more than one comorbidity, totals may not sum to 100\%."}\NormalTok{,}
    \AttributeTok{general\_title =} \StringTok{""}\NormalTok{,}
    \AttributeTok{footnote\_as\_chunk =} \ConstantTok{TRUE}
\NormalTok{  ) }
\end{Highlighting}
\end{Shaded}

\begin{table}
\centering
\begin{tabular}[t]{lcc}
\toprule
\multicolumn{1}{c}{} & \multicolumn{2}{c}{Overall (N = 185)} \\
\cmidrule(l{3pt}r{3pt}){2-3}
\multicolumn{1}{c}{Variables} & \multicolumn{1}{c}{N} & \multicolumn{1}{c}{Stat}\\
\midrule
\textbf{Comorbidities1¹} & 91 & 91 (100%)\\
\hspace{1em}Diabetes &  & 33 (36%)\\
\hspace{1em}Hypertension &  & 35 (38%)\\
\hspace{1em}COPD &  & 41 (45%)\\
\textbf{Comorbidities2¹} & 66 & \\
\addlinespace
\hspace{1em}CKD &  & 39 (59%)\\
\hspace{1em}CAD &  & 38 (58%)\\
\bottomrule
\multicolumn{3}{l}{\rule{0pt}{1em}¹Patients could present with more than one comorbidity, totals may not sum to 100\%.}\\
\end{tabular}
\end{table}

\hypertarget{table_one}{%
\subsection{\texorpdfstring{\texttt{table\_one()}}{table\_one()}}\label{table_one}}

\begin{itemize}
\tightlist
\item
  \texttt{table\_one()} function will create a clean report-ready kable table.
\item
  I would suggest reviewing the help file for \texttt{table\_one()} to explore the additional functionality and default arguments.
\end{itemize}

\begin{Shaded}
\begin{Highlighting}[]
\NormalTok{Comorbidities  }\OtherTok{\textless{}{-}}\NormalTok{ cardio\_data }\SpecialCharTok{\%\textgreater{}\%} \FunctionTok{select}\NormalTok{(Diabetes}\SpecialCharTok{:}\NormalTok{CAD) }\SpecialCharTok{\%\textgreater{}\%} \FunctionTok{names}\NormalTok{()}

\NormalTok{work\_d }\OtherTok{\textless{}{-}}\NormalTok{ cardio\_data }\SpecialCharTok{\%\textgreater{}\%} 
  \FunctionTok{mutate}\NormalTok{(}\AttributeTok{SurgeryType =} \FunctionTok{factor\_order}\NormalTok{(SurgeryType)) }\SpecialCharTok{\%\textgreater{}\%} 
  \FunctionTok{check\_box\_convert}\NormalTok{(}\AttributeTok{check\_box\_cols =}\NormalTok{ Comorbidities,}\AttributeTok{title =} \StringTok{"Comorbidities¹"}\NormalTok{) }

\FunctionTok{table\_one}\NormalTok{(}\AttributeTok{df =}\NormalTok{ work\_d , }
          \AttributeTok{group =}\NormalTok{ Sex,  }
          \AttributeTok{caption =}  \StringTok{"Summary table overall and stratified by sex"}\NormalTok{,}
          \AttributeTok{include\_overall =} \StringTok{"all"}\NormalTok{,}
          \AttributeTok{Check\_box =}\NormalTok{ Comorbidities,}
          \AttributeTok{Check\_box\_title =} \StringTok{"Comorbidities¹"}\NormalTok{)}\SpecialCharTok{\%\textgreater{}\%}
\NormalTok{  kableExtra}\SpecialCharTok{::}\FunctionTok{footnote}\NormalTok{(}
    \AttributeTok{general =} \StringTok{"¹Patients could present with more than one comorbidity, totals may not sum to 100\%."}\NormalTok{,}
    \AttributeTok{general\_title =} \StringTok{""}\NormalTok{,}
    \AttributeTok{footnote\_as\_chunk =} \ConstantTok{TRUE}\NormalTok{)}
\end{Highlighting}
\end{Shaded}

\begin{table}
\centering
\caption{\label{tab:unnamed-chunk-3}Summary table overall and stratified by sex}
\centering
\begin{tabular}[t]{lccccccr}
\toprule
\multicolumn{1}{c}{} & \multicolumn{2}{c}{Overall (N = 185)} & \multicolumn{2}{c}{Female (N = 87)} & \multicolumn{2}{c}{Male (N = 98)} & \multicolumn{1}{c}{} \\
\cmidrule(l{3pt}r{3pt}){2-3} \cmidrule(l{3pt}r{3pt}){4-5} \cmidrule(l{3pt}r{3pt}){6-7}
\multicolumn{1}{c}{Variables} & \multicolumn{1}{c}{N} & \multicolumn{1}{c}{Stat} & \multicolumn{1}{c}{N} & \multicolumn{1}{c}{Stat} & \multicolumn{1}{c}{N} & \multicolumn{1}{c}{Stat} & \multicolumn{1}{c}{*P*-value}\\
\midrule
\textbf{Age} & 185 & 51 (38 – 69) & 87 & 52 (40 – 71) & 98 & 50 (37 – 63) & 0.58\\
\textbf{NYHA} & 135 &  & 67 &  & 68 &  & 0.39\\
\hspace{1em}I &  & 31 (23%) &  & 16 (24%) &  & 15 (22%) & \\
\hspace{1em}II &  & 0 (0%) &  & 0 (0%) &  & 0 (0%) & \\
\hspace{1em}III &  & 52 (39%) &  & 22 (33%) &  & 30 (44%) & \\
\addlinespace
\hspace{1em}IV &  & 52 (39%) &  & 29 (43%) &  & 23 (34%) & \\
\textbf{EjectionFraction} & 185 & 44.7 (31.8 – 56.0) & 87 & 44.7 (32.1 – 56.0) & 98 & 44.9 (32.0 – 56.0) & 0.92\\
\textbf{BMI} & 185 & 26.6 (22.8 – 30.2) & 87 & 25.5 (22.1 – 30.1) & 98 & 27.4 (23.5 – 30.3) & 0.193\\
\textbf{HeartRate} & 185 & 83 (67 – 99) & 87 & 84 (67 – 98) & 98 & 83 (67 – 98) & 0.96\\
\textbf{SBP} & 185 & 133 (110 – 154) & 87 & 129 (110 – 153) & 98 & 135 (110 – 156) & 0.60\\
\addlinespace
\textbf{DBP} & 185 & 80 (70 – 92) & 87 & 78 (69 – 89) & 98 & 84 (72 – 94) & 0.019\\
\textbf{SmokingStatus} & 185 &  & 87 &  & 98 &  & 0.37\\
\hspace{1em}Current &  & 61 (33%) &  & 33 (38%) &  & 28 (29%) & \\
\hspace{1em}Former &  & 69 (37%) &  & 29 (33%) &  & 40 (41%) & \\
\hspace{1em}Never &  & 55 (30%) &  & 25 (29%) &  & 30 (31%) & \\
\addlinespace
\textbf{HistoryMI} & 185 & 89 (48%) & 87 & 41 (47%) & 98 & 48 (49%) & 0.88\\
\textbf{Comorbidities¹} & 128 &  & 61 &  & 67 &  & \\
\hspace{1em}Diabetes &  & 33 (26%) &  & 18 (30%) &  & 15 (22%) & 0.42\\
\hspace{1em}Hypertension &  & 35 (27%) &  & 14 (23%) &  & 21 (31%) & 0.33\\
\hspace{1em}COPD &  & 41 (32%) &  & 20 (33%) &  & 21 (31%) & 1.00\\
\addlinespace
\hspace{1em}CKD &  & 39 (30%) &  & 17 (28%) &  & 22 (33%) & 0.57\\
\hspace{1em}CAD &  & 38 (30%) &  & 17 (28%) &  & 21 (31%) & 0.70\\
\textbf{SurgeryType} & 185 &  & 87 &  & 98 &  & 0.053\\
\hspace{1em}Mitral replacement &  & 77 (42%) &  & 42 (48%) &  & 35 (36%) & \\
\hspace{1em}CABG &  & 50 (27%) &  & 23 (26%) &  & 27 (28%) & \\
\addlinespace
\hspace{1em}Mitral repair &  & 20 (11%) &  & 7 (8%) &  & 13 (13%) & \\
\hspace{1em}Aortic repair &  & 9 (5%) &  & 4 (5%) &  & 5 (5%) & \\
\hspace{1em}LVAD &  & 8 (4%) &  & 0 (0%) &  & 8 (8%) & \\
\hspace{1em}Transplant &  & 8 (4%) &  & 6 (7%) &  & 2 (2%) & \\
\hspace{1em}Aortic replacement &  & 7 (4%) &  & 3 (3%) &  & 4 (4%) & \\
\addlinespace
\hspace{1em}Maze &  & 6 (3%) &  & 2 (2%) &  & 4 (4%) & \\
\bottomrule
\multicolumn{8}{l}{\rule{0pt}{1em}¹Patients could present with more than one comorbidity, totals may not sum to 100\%.}\\
\end{tabular}
\end{table}

\hypertarget{descriptive-survival-functions}{%
\section{Descriptive Survival functions}\label{descriptive-survival-functions}}

\hypertarget{construct_surv_cmprisk_var}{%
\subsection{\texorpdfstring{\texttt{construct\_surv\_cmprisk\_var()}}{construct\_surv\_cmprisk\_var()}}\label{construct_surv_cmprisk_var}}

\begin{itemize}
\item
  The previous fanetc functions \texttt{construct\_surv\_var()} and \texttt{construct\_cmprisk\_var()} have been combined into a single function for creating survival and CIF datasets.
\item
  You can specify the time units as ``days'' (default), ``weeks'', ``months'', or ``years''.
\item
  Administrative censoring can also be included with the adm\_cnr\_time argument and does not require an additional function.
\item
  The function assumes CIF data when a competing event is included, but this can be overridden using the cmprisk argument.
\end{itemize}

\begin{Shaded}
\begin{Highlighting}[]
\DocumentationTok{\#\# For survival}
\NormalTok{survival\_data }\OtherTok{\textless{}{-}}  \FunctionTok{construct\_surv\_cmprisk\_var}\NormalTok{(cardio\_data,}
                           \AttributeTok{patid =}\NormalTok{ PatientID,}
                           \AttributeTok{idx\_dt =}\NormalTok{ SurgeryDate,}
                           \AttributeTok{evt\_dt =}\NormalTok{ DeathDate,}
                           \AttributeTok{end\_dt =}\NormalTok{ LastVisitDate,}
                           \AttributeTok{append =} \ConstantTok{TRUE}\NormalTok{,}
                           \AttributeTok{units =} \StringTok{"months"}\NormalTok{,}
                           \AttributeTok{adm\_cnr\_time =} \DecValTok{24}\NormalTok{)}

\CommentTok{\# For competing risks}
\NormalTok{cmp\_risk\_data }\OtherTok{\textless{}{-}} \FunctionTok{construct\_surv\_cmprisk\_var}\NormalTok{(cardio\_data,}
                           \AttributeTok{patid =}\NormalTok{ PatientID,}
                           \AttributeTok{idx\_dt =}\NormalTok{ SurgeryDate,}
                           \AttributeTok{evt\_dt =}\NormalTok{ TransplantDate,}
                           \AttributeTok{end\_dt =}\NormalTok{ LastVisitDate,}
                           \AttributeTok{other\_dt =}\NormalTok{ DeathDate,}
                           \AttributeTok{append =} \ConstantTok{TRUE}\NormalTok{,}
                           \AttributeTok{units =} \StringTok{"months"}\NormalTok{,}
                           \AttributeTok{adm\_cnr\_time =} \DecValTok{24}\NormalTok{)}
\end{Highlighting}
\end{Shaded}

\hypertarget{estimate_cif_km}{%
\subsection{\texorpdfstring{\texttt{estimate\_cif\_km()}}{estimate\_cif\_km()}}\label{estimate_cif_km}}

\begin{itemize}
\tightlist
\item
  The previous fanetc functions \texttt{estimate\_km()} and \texttt{estimate\_cif()} have been combined into a single function for estimating KM and CIF.
\item
  When evt is a numeric column, it will produce the KM estimates.
\item
  When evt is a factor it will produce CIF estimates.
\end{itemize}

\begin{Shaded}
\begin{Highlighting}[]
\DocumentationTok{\#\# For survival}
\NormalTok{KM }\OtherTok{\textless{}{-}}  \FunctionTok{construct\_surv\_cmprisk\_var}\NormalTok{(cardio\_data,}
                           \AttributeTok{patid =}\NormalTok{ PatientID,}
                           \AttributeTok{idx\_dt =}\NormalTok{ SurgeryDate,}
                           \AttributeTok{evt\_dt =}\NormalTok{ DeathDate,}
                           \AttributeTok{end\_dt =}\NormalTok{ LastVisitDate,}
                           \AttributeTok{append =} \ConstantTok{TRUE}\NormalTok{,}
                           \AttributeTok{units =} \StringTok{"months"}\NormalTok{,}
                           \AttributeTok{adm\_cnr\_time =} \DecValTok{24}\NormalTok{) }\SpecialCharTok{\%\textgreater{}\%} 
\FunctionTok{estimate\_cif\_km}\NormalTok{(}\AttributeTok{evt =}\NormalTok{ evt,}\AttributeTok{evt\_time =}\NormalTok{ evt\_time)}



\CommentTok{\# For competing risks}
\NormalTok{CIF }\OtherTok{\textless{}{-}} \FunctionTok{construct\_surv\_cmprisk\_var}\NormalTok{(cardio\_data,}
                           \AttributeTok{patid =}\NormalTok{ PatientID,}
                           \AttributeTok{idx\_dt =}\NormalTok{ SurgeryDate,}
                           \AttributeTok{evt\_dt =}\NormalTok{ TransplantDate,}
                           \AttributeTok{end\_dt =}\NormalTok{ LastVisitDate,}
                           \AttributeTok{other\_dt =}\NormalTok{ DeathDate,}
                           \AttributeTok{append =} \ConstantTok{TRUE}\NormalTok{,}
                           \AttributeTok{units =} \StringTok{"months"}\NormalTok{,}
                           \AttributeTok{adm\_cnr\_time =} \DecValTok{24}\NormalTok{)}\SpecialCharTok{\%\textgreater{}\%} 
\FunctionTok{estimate\_cif\_km}\NormalTok{(}\AttributeTok{evt =}\NormalTok{ evt,}\AttributeTok{evt\_time =}\NormalTok{ evt\_time)}
\end{Highlighting}
\end{Shaded}

\hypertarget{summarize_km}{%
\subsection{\texorpdfstring{\texttt{summarize\_km()}}{summarize\_km()}}\label{summarize_km}}

\begin{itemize}
\tightlist
\item
  This function will produce a clean report-ready kable table displaying the results from the KM model.
\end{itemize}

\begin{Shaded}
\begin{Highlighting}[]
 \DocumentationTok{\#\# Example without a covariate}

 \FunctionTok{construct\_surv\_cmprisk\_var}\NormalTok{(cardio\_data,}
                           \AttributeTok{patid =}\NormalTok{ PatientID,}
                           \AttributeTok{idx\_dt =}\NormalTok{ SurgeryDate,}
                           \AttributeTok{evt\_dt =}\NormalTok{ DeathDate,}
                           \AttributeTok{end\_dt =}\NormalTok{ LastVisitDate,}
                           \AttributeTok{append =} \ConstantTok{TRUE}\NormalTok{,}
                           \AttributeTok{units =} \StringTok{"months"}\NormalTok{,}
                           \AttributeTok{adm\_cnr\_time =} \DecValTok{24}\NormalTok{) }\SpecialCharTok{\%\textgreater{}\%} 
\FunctionTok{estimate\_cif\_km}\NormalTok{(}\AttributeTok{evt =}\NormalTok{ evt,}\AttributeTok{evt\_time =}\NormalTok{ evt\_time) }\SpecialCharTok{\%\textgreater{}\%}
  \FunctionTok{summarize\_km}\NormalTok{(}\AttributeTok{overall\_label =} \StringTok{"All patients"}\NormalTok{,}\AttributeTok{time\_lab =} \StringTok{"Time since surgery (months)"}\NormalTok{,}\AttributeTok{caption =} \StringTok{"Overall Survival for all patients"}\NormalTok{)}
\end{Highlighting}
\end{Shaded}

\begin{table}
\centering
\caption{\label{tab:unnamed-chunk-6}Overall Survival for all patients}
\centering
\begin{tabular}[t]{lc}
\toprule
\multicolumn{1}{c}{Time since surgery (months)} & \multicolumn{1}{c}{All patients}\\
\midrule
0 & 100.0% [100.0%, 100.0%]\\
5 & 84.4% [78.1%, 89.1%]\\
10 & 75.8% [68.2%, 81.7%]\\
15 & 74.0% [66.1%, 80.2%]\\
20 & 74.0% [66.1%, 80.2%]\\
\bottomrule
\end{tabular}
\end{table}

\begin{Shaded}
\begin{Highlighting}[]
\DocumentationTok{\#\# Example with a covariate}

 \FunctionTok{construct\_surv\_cmprisk\_var}\NormalTok{(cardio\_data,}
                           \AttributeTok{patid =}\NormalTok{ PatientID,}
                           \AttributeTok{idx\_dt =}\NormalTok{ SurgeryDate,}
                           \AttributeTok{evt\_dt =}\NormalTok{ DeathDate,}
                           \AttributeTok{end\_dt =}\NormalTok{ LastVisitDate,}
                           \AttributeTok{append =} \ConstantTok{TRUE}\NormalTok{,}
                           \AttributeTok{units =} \StringTok{"months"}\NormalTok{,}
                           \AttributeTok{adm\_cnr\_time =} \DecValTok{24}\NormalTok{) }\SpecialCharTok{\%\textgreater{}\%} 
\FunctionTok{estimate\_cif\_km}\NormalTok{(}\AttributeTok{evt =}\NormalTok{ evt,}\AttributeTok{evt\_time =}\NormalTok{ evt\_time,Sex)  }\SpecialCharTok{\%\textgreater{}\%}
  \FunctionTok{summarize\_km}\NormalTok{(}\AttributeTok{overall\_label =} \StringTok{"All patients"}\NormalTok{,}\AttributeTok{time\_lab =} \StringTok{"Time since surgery (months)"}\NormalTok{,}\AttributeTok{caption =} \StringTok{"Overall Survival by sex"}\NormalTok{)}
\end{Highlighting}
\end{Shaded}

\begin{table}
\centering
\caption{\label{tab:unnamed-chunk-6}Overall Survival by sex}
\centering
\begin{tabular}[t]{lcc}
\toprule
\multicolumn{1}{c}{Time since surgery (months)} & \multicolumn{1}{c}{Female} & \multicolumn{1}{c}{Male}\\
\midrule
0 & 100.0% [100.0%, 100.0%] & 100.0% [100.0%, 100.0%]\\
5 & 92.8% [84.7%, 96.7%] & 76.8% [66.6%, 84.2%]\\
10 & 82.4% [71.5%, 89.5%] & 69.7% [58.5%, 78.4%]\\
15 & 79.0% [67.4%, 86.9%] & 69.7% [58.5%, 78.4%]\\
20 & 79.0% [67.4%, 86.9%] & 69.7% [58.5%, 78.4%]\\
\bottomrule
\end{tabular}
\end{table}

\hypertarget{summarize_cif}{%
\subsection{\texorpdfstring{\texttt{summarize\_cif()}}{summarize\_cif()}}\label{summarize_cif}}

\begin{itemize}
\tightlist
\item
  This function will produce a clean report-ready kable table displaying the results from the CIF model.
\item
  The argument evt\_type is used to specify which events to report.
\item
  The argument labels is used to specify the column names for each event type.
\end{itemize}

\begin{Shaded}
\begin{Highlighting}[]
\DocumentationTok{\#\# Presenting one event for all patients }

\FunctionTok{construct\_surv\_cmprisk\_var}\NormalTok{(cardio\_data,}
                           \AttributeTok{patid =}\NormalTok{ PatientID,}
                           \AttributeTok{idx\_dt =}\NormalTok{ SurgeryDate,}
                           \AttributeTok{evt\_dt =}\NormalTok{ TransplantDate,}
                           \AttributeTok{end\_dt =}\NormalTok{ LastVisitDate,}
                           \AttributeTok{other\_dt =}\NormalTok{ DeathDate,}
                           \AttributeTok{append =} \ConstantTok{TRUE}\NormalTok{,}
                           \AttributeTok{units =} \StringTok{"months"}\NormalTok{,}
                           \AttributeTok{adm\_cnr\_time =} \DecValTok{24}\NormalTok{)}\SpecialCharTok{\%\textgreater{}\%} 
\FunctionTok{estimate\_cif\_km}\NormalTok{(}\AttributeTok{evt =}\NormalTok{ evt,}\AttributeTok{evt\_time =}\NormalTok{ evt\_time)}\SpecialCharTok{\%\textgreater{}\%}
  \FunctionTok{summarize\_cif}\NormalTok{(}\AttributeTok{time\_lab =} \StringTok{"Time since surgery (months)"}\NormalTok{,}\AttributeTok{evt\_type =} \DecValTok{1}\NormalTok{,}\AttributeTok{caption =} \StringTok{"Time to transplant for all patients"}\NormalTok{,}
  \AttributeTok{labels =} \FunctionTok{c}\NormalTok{(}\StringTok{\textasciigrave{}}\AttributeTok{(s0)}\StringTok{\textasciigrave{}} \OtherTok{=} \StringTok{"Event free"}\NormalTok{, }\StringTok{\textasciigrave{}}\AttributeTok{1}\StringTok{\textasciigrave{}} \OtherTok{=} \StringTok{"Transplant"}\NormalTok{, }\StringTok{\textasciigrave{}}\AttributeTok{2}\StringTok{\textasciigrave{}} \OtherTok{=} \StringTok{"Death"}\NormalTok{))}
\end{Highlighting}
\end{Shaded}

\begin{table}
\centering
\caption{\label{tab:unnamed-chunk-7}Time to transplant for all patients}
\centering
\begin{tabular}[t]{lc}
\toprule
\multicolumn{1}{c}{Time since surgery (months)} & \multicolumn{1}{c}{Overall}\\
\midrule
0 & 0.0% [0.0%, 0.0%]\\
5 & 4.0% [1.9%, 8.3%]\\
10 & 7.2% [4.2%, 12.4%]\\
15 & 11.8% [7.6%, 18.3%]\\
20 & 17.4% [12.0%, 25.3%]\\
\bottomrule
\end{tabular}
\end{table}

\begin{Shaded}
\begin{Highlighting}[]
\DocumentationTok{\#\# Presenting all events for all patients }

\FunctionTok{construct\_surv\_cmprisk\_var}\NormalTok{(cardio\_data,}
                           \AttributeTok{patid =}\NormalTok{ PatientID,}
                           \AttributeTok{idx\_dt =}\NormalTok{ SurgeryDate,}
                           \AttributeTok{evt\_dt =}\NormalTok{ TransplantDate,}
                           \AttributeTok{end\_dt =}\NormalTok{ LastVisitDate,}
                           \AttributeTok{other\_dt =}\NormalTok{ DeathDate,}
                           \AttributeTok{append =} \ConstantTok{TRUE}\NormalTok{,}
                           \AttributeTok{units =} \StringTok{"months"}\NormalTok{,}
                           \AttributeTok{adm\_cnr\_time =} \DecValTok{24}\NormalTok{)}\SpecialCharTok{\%\textgreater{}\%} 
\FunctionTok{estimate\_cif\_km}\NormalTok{(}\AttributeTok{evt =}\NormalTok{ evt,}\AttributeTok{evt\_time =}\NormalTok{ evt\_time)}\SpecialCharTok{\%\textgreater{}\%}
  \FunctionTok{summarize\_cif}\NormalTok{(}\AttributeTok{time\_lab =} \StringTok{"Time since surgery (months)"}\NormalTok{,}\AttributeTok{caption =} \StringTok{"Time to transplant for all patients"}\NormalTok{,}
  \AttributeTok{labels =} \FunctionTok{c}\NormalTok{(}\StringTok{\textasciigrave{}}\AttributeTok{(s0)}\StringTok{\textasciigrave{}} \OtherTok{=} \StringTok{"Event free"}\NormalTok{, }\StringTok{\textasciigrave{}}\AttributeTok{1}\StringTok{\textasciigrave{}} \OtherTok{=} \StringTok{"Transplant"}\NormalTok{, }\StringTok{\textasciigrave{}}\AttributeTok{2}\StringTok{\textasciigrave{}} \OtherTok{=} \StringTok{"Death"}\NormalTok{))}
\end{Highlighting}
\end{Shaded}

\begin{table}
\centering
\caption{\label{tab:unnamed-chunk-7}Time to transplant for all patients}
\centering
\begin{tabular}[t]{lccc}
\toprule
\multicolumn{1}{c}{Time since surgery (months)} & \multicolumn{1}{c}{Event free} & \multicolumn{1}{c}{Transplant} & \multicolumn{1}{c}{Death}\\
\midrule
0 & 100.0% [0.0%, 0.0%] & 0.0% [0.0%, 0.0%] & 0.0% [0.0%, 0.0%]\\
5 & 81.3% [75.8%, 87.3%] & 4.0% [1.9%, 8.3%] & 14.7% [10.3%, 20.9%]\\
10 & 70.2% [63.5%, 77.6%] & 7.2% [4.2%, 12.4%] & 22.6% [17.0%, 30.0%]\\
15 & 64.1% [57.0%, 72.1%] & 11.8% [7.6%, 18.3%] & 24.1% [18.3%, 31.6%]\\
20 & 58.5% [50.9%, 67.2%] & 17.4% [12.0%, 25.3%] & 24.1% [18.3%, 31.6%]\\
\bottomrule
\end{tabular}
\end{table}

\begin{Shaded}
\begin{Highlighting}[]
\DocumentationTok{\#\# Presenting only the events by a covariate}

\FunctionTok{construct\_surv\_cmprisk\_var}\NormalTok{(cardio\_data,}
                           \AttributeTok{patid =}\NormalTok{ PatientID,}
                           \AttributeTok{idx\_dt =}\NormalTok{ SurgeryDate,}
                           \AttributeTok{evt\_dt =}\NormalTok{ TransplantDate,}
                           \AttributeTok{end\_dt =}\NormalTok{ LastVisitDate,}
                           \AttributeTok{other\_dt =}\NormalTok{ DeathDate,}
                           \AttributeTok{append =} \ConstantTok{TRUE}\NormalTok{,}
                           \AttributeTok{units =} \StringTok{"months"}\NormalTok{,}
                           \AttributeTok{adm\_cnr\_time =} \DecValTok{24}\NormalTok{)}\SpecialCharTok{\%\textgreater{}\%} 
\FunctionTok{estimate\_cif\_km}\NormalTok{(}\AttributeTok{evt =}\NormalTok{ evt,}\AttributeTok{evt\_time =}\NormalTok{ evt\_time, }\AttributeTok{group =}\NormalTok{ Sex)}\SpecialCharTok{\%\textgreater{}\%}
  \FunctionTok{summarize\_cif}\NormalTok{(}\AttributeTok{time\_lab =} \StringTok{"Time since surgery (months)"}\NormalTok{,}\AttributeTok{evt\_type =} \DecValTok{1}\NormalTok{,}\AttributeTok{caption =} \StringTok{"Time to transplant for all patients"}\NormalTok{,}
  \AttributeTok{labels =} \FunctionTok{c}\NormalTok{(}\StringTok{\textasciigrave{}}\AttributeTok{(s0)}\StringTok{\textasciigrave{}} \OtherTok{=} \StringTok{"Event free"}\NormalTok{, }\StringTok{\textasciigrave{}}\AttributeTok{1}\StringTok{\textasciigrave{}} \OtherTok{=} \StringTok{"Transplant"}\NormalTok{, }\StringTok{\textasciigrave{}}\AttributeTok{2}\StringTok{\textasciigrave{}} \OtherTok{=} \StringTok{"Death"}\NormalTok{))}
\end{Highlighting}
\end{Shaded}

\begin{table}
\centering
\caption{\label{tab:unnamed-chunk-7}Time to transplant for all patients}
\centering
\begin{tabular}[t]{lcc}
\toprule
\multicolumn{1}{c}{Time since surgery (months)} & \multicolumn{1}{c}{Female} & \multicolumn{1}{c}{Male}\\
\midrule
0 & 0.0% [0.0%, 0.0%] & 0.0% [0.0%, 0.0%]\\
5 & 3.6% [1.2%, 11.1%] & 4.3% [1.7%, 11.3%]\\
10 & 7.7% [3.5%, 16.6%] & 6.7% [3.1%, 14.7%]\\
15 & 13.7% [7.7%, 24.5%] & 9.9% [5.1%, 19.3%]\\
20 & 15.3% [8.8%, 26.5%] & 20.8% [12.5%, 34.7%]\\
\bottomrule
\end{tabular}
\end{table}

\begin{Shaded}
\begin{Highlighting}[]
\DocumentationTok{\#\# Presenting all events by a covariate }

\FunctionTok{construct\_surv\_cmprisk\_var}\NormalTok{(cardio\_data,}
                           \AttributeTok{patid =}\NormalTok{ PatientID,}
                           \AttributeTok{idx\_dt =}\NormalTok{ SurgeryDate,}
                           \AttributeTok{evt\_dt =}\NormalTok{ TransplantDate,}
                           \AttributeTok{end\_dt =}\NormalTok{ LastVisitDate,}
                           \AttributeTok{other\_dt =}\NormalTok{ DeathDate,}
                           \AttributeTok{append =} \ConstantTok{TRUE}\NormalTok{,}
                           \AttributeTok{units =} \StringTok{"months"}\NormalTok{,}
                           \AttributeTok{adm\_cnr\_time =} \DecValTok{24}\NormalTok{)}\SpecialCharTok{\%\textgreater{}\%} 
\FunctionTok{estimate\_cif\_km}\NormalTok{(}\AttributeTok{evt =}\NormalTok{ evt,}\AttributeTok{evt\_time =}\NormalTok{ evt\_time, }\AttributeTok{group =}\NormalTok{ Sex)}\SpecialCharTok{\%\textgreater{}\%}
  \FunctionTok{summarize\_cif}\NormalTok{(}\AttributeTok{time\_lab =} \StringTok{"Time since surgery (months)"}\NormalTok{,}\AttributeTok{caption =} \StringTok{"Time to transplant for all patients"}\NormalTok{,}
  \AttributeTok{labels =} \FunctionTok{c}\NormalTok{(}\StringTok{\textasciigrave{}}\AttributeTok{(s0)}\StringTok{\textasciigrave{}} \OtherTok{=} \StringTok{"Event free"}\NormalTok{, }\StringTok{\textasciigrave{}}\AttributeTok{1}\StringTok{\textasciigrave{}} \OtherTok{=} \StringTok{"Transplant"}\NormalTok{, }\StringTok{\textasciigrave{}}\AttributeTok{2}\StringTok{\textasciigrave{}} \OtherTok{=} \StringTok{"Death"}\NormalTok{))}
\end{Highlighting}
\end{Shaded}

\begin{table}
\centering
\caption{\label{tab:unnamed-chunk-7}Time to transplant for all patients}
\centering
\begin{tabular}[t]{lcccccc}
\toprule
\multicolumn{1}{c}{} & \multicolumn{2}{c}{Event free} & \multicolumn{2}{c}{Transplant} & \multicolumn{2}{c}{Death} \\
\cmidrule(l{3pt}r{3pt}){2-3} \cmidrule(l{3pt}r{3pt}){4-5} \cmidrule(l{3pt}r{3pt}){6-7}
\multicolumn{1}{c}{Time since surgery (months)} & \multicolumn{1}{c}{Female} & \multicolumn{1}{c}{Male} & \multicolumn{1}{c}{Female} & \multicolumn{1}{c}{Male} & \multicolumn{1}{c}{Female} & \multicolumn{1}{c}{Male}\\
\midrule
0 & 100.0% [0.0%, 0.0%] & 100.0% [0.0%, 0.0%] & 0.0% [0.0%, 0.0%] & 0.0% [0.0%, 0.0%] & 0.0% [0.0%, 0.0%] & 0.0% [0.0%, 0.0%]\\
5 & 89.3% [82.9%, 96.2%] & 74.0% [65.6%, 83.6%] & 3.6% [1.2%, 11.1%] & 4.3% [1.7%, 11.3%] & 7.1% [3.3%, 15.3%] & 21.7% [14.7%, 32.0%]\\
10 & 75.9% [66.9%, 86.0%] & 65.1% [55.7%, 75.9%] & 7.7% [3.5%, 16.6%] & 6.7% [3.1%, 14.7%] & 16.5% [10.0%, 27.1%] & 28.2% [20.1%, 39.5%]\\
15 & 67.0% [57.0%, 78.6%] & 61.9% [52.3%, 73.3%] & 13.7% [7.7%, 24.5%] & 9.9% [5.1%, 19.3%] & 19.3% [12.2%, 30.5%] & 28.2% [20.1%, 39.5%]\\
20 & 65.4% [55.3%, 77.3%] & 51.0% [40.1%, 64.9%] & 15.3% [8.8%, 26.5%] & 20.8% [12.5%, 34.7%] & 19.3% [12.2%, 30.5%] & 28.2% [20.1%, 39.5%]\\
\bottomrule
\end{tabular}
\end{table}

\hypertarget{show_surv}{%
\subsection{\texorpdfstring{\texttt{show\_surv()}}{show\_surv()}}\label{show_surv}}

\hypertarget{show_cif}{%
\subsection{\texorpdfstring{\texttt{show\_cif()}}{show\_cif()}}\label{show_cif}}

\begin{itemize}
\tightlist
\item
  By default the \texttt{show\_surv()} and \texttt{show\_cif()} function print the number at risk table just by adding the function \texttt{grid.draw()}
\end{itemize}

\begin{Shaded}
\begin{Highlighting}[]
\CommentTok{\# relpase\_df \%\textgreater{}\% }
\CommentTok{\#   estimate\_cif(evt\_time = evt\_time,evt = evt,group = Treatment) \%\textgreater{}\%  }
\CommentTok{\#   show\_cif (x\_lab = "Years since first treatment",}
\CommentTok{\#          y\_lab =  "Cumulative incidence of relapse (\%)",print\_fig=FALSE) \%\textgreater{}\%  grid.draw()}
\end{Highlighting}
\end{Shaded}

)

\end{document}
